\documentclass[11pt]{article}

\usepackage{fancybox}
\usepackage{graphicx}
\usepackage{pdfpages}
\usepackage{color}
\usepackage{epstopdf}
\usepackage[margin=1in, vmargin=1in]{geometry}
\usepackage{amsmath}
\usepackage{amsfonts}
\usepackage{float}
\usepackage{listings}
\usepackage{verbatim}
\usepackage{booktabs}
\usepackage{tabularx}
\usepackage{longtable}
\usepackage{amsmath}
\usepackage{movie15}
\usepackage{hyperref}
\usepackage{subcaption}
\usepackage{enumerate}



\sloppy
\definecolor{lightgray}{gray}{0.5}
\setlength{\parindent}{20pt}


\newcommand{\field}[1]{\mathbb{#1}}
\def\aa{\}}



\begin{document}




\begin{titlepage}    
\begin{flushright}
\vspace*{.5in}



\Large{ \sc Micros, George\\ }


\vspace*{.5in}

\Large{Assignment \# 2\\}


\vspace{1in}
\Large{\sc CS 463: Cryptography for Cybersecurity\\ Instructor: Ravi Mukkamala, Ph.D \\ Fall 2014 }




\vspace{.5cm}




\vspace {7cm}


\end{flushright}
\end{titlepage}



%%%%%%%%%%%%%%%%%%%%%%%%%%%%%%%%%%%%%%%%%%%%%%%%%%%%%%%%%%%
\section{\sc Question 1}

\subsection{Part a}

\begin{flushleft}
$(234*145) \mod 10$

$(234 \mod 10)*(145 \mod 10) \mod 10$

$ 4*5 \mod 10 \equiv 0 $

\end{flushleft}


\subsection{Part b}

\begin{flushleft}
$ 7*\frac{4}{11} \mod 10$

$ 7 * \frac{4 + 4*10}{11} \mod 10$

$ 7*4 \mod 10  \equiv  28 \mod 10 \equiv8 $

\end{flushleft}

\subsection{Part c}

\begin{flushleft}
$ 8^{202} * 7^{103}  \mod 10$

$ (8^{202 \mod 100} \mod 10)*( 7^{103 \mod 48} \mod 10) \mod 10 $

$ (64 \mod 10)*(7^{34 \mod 16} \mod 10) \mod 10 $

$ 4 * ( 7^2 \mod 10 ) \mod 10 \equiv 4*4 \mod 10  \equiv 6 $


\end{flushleft}

%%%%%%%%%%%%%%%%%%%%%%%%%%%%%%%%%%%%%%%%%%%%%%%%%%%%%%%%%%%
\section{\sc Question 2}

\subsection{Part a}

\begin{flushleft}
$ \field{Z}_{12}$ = \{0,1,2,3,4,5,6,7,8,9,10,11\} 

$ \field{Z*}_{12} $= \{1,5,7,11\} 
\end{flushleft}

\subsection{Part b}

\begin{flushleft}
$ 7^2 \mod 12 \equiv 1 $

order(7) in $\field{Z*}_{12} = 2$

\end{flushleft}


\subsection{Part c}

\begin{flushleft}
The multiplicative inverse of 5 in $\field{Z*}_{12}$ is itself

$ 5*5 \mod 12 \equiv 1 $



\end{flushleft}
%%%%%%%%%%%%%%%%%%%%%%%%%%%%%%%%%%%%%%%%%%%%%%%%%%%%%%%%%%%



\end{document}


